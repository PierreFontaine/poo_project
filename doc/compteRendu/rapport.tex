\documentclass[12pt]{report}
\usepackage[utf8]{inputenc}
\usepackage[french]{babel}
\usepackage{amsmath,amsfonts,amssymb,tikz,graphicx,geometry,tkz-tab,pgfplots,float}

\title{Rapport de Projet en Programmation Orientée Objet C++}
\author{Sallio Romane and Fontaine Pierre}
\begin{document}
  \maketitle
  \newpage
  \tableofcontents
  \begin{abstract}
    Ce projet à pour objectif de ré-introduire les notions vu en cours de programmation orientée objet et introduites lors des TPs. Pour cela nous avons donc créé un Dashboard composée de 5 widgets (météo,actualité,todolist,convertisseur et horloge).  
  \end{abstract}
  \newpage
  \part{Introduction}
    Pour ce projet nous avons choisis d'implémenter un "Dashboard", cela se présente sous forme d'une fenêtre graphique qui est développer en C++ grâce au framework QT largement utilisé dans le monde professionel de l'informatique.\\
    Il implémente une todolist, la recherche de la météo grâce à une API, la recherche d'actualité grâce à un flux RSS, une horloge et un convertisseur de vitesse et de température.
  \part{Widgets}
  \section{Menu}
    Le menu est composé d'un main.cpp, mainwindow.cpp/hpp, module.cpp/hpp.\\
    Le main.cpp permet l'affichage de la fentere principal et la création du dossier dashboard.\\
    La classe MainWindow permet la création des différents modules et de les afficher.\\
    La classe Module est le modèle pour les classes ToDoListModule, MeteoModule.. qui en dérivent.\\
  \section{ToDoList}
    ***La todolist est composée de todolistajout.cpp/hpp, todolistdata.cpp/hpp, todolistdisplay.cpp/hpp, todolistmodule.hpp/hpp.\\
    La classe ToDoListModule est une classe qui hérite de QWidget, elle permet d'afficher le calendrier et d'ouvirir la fenêtre ToDoListAjout lorsque l'on double cliques sur une date supérieure ou égale à la date actuelle.\\
    ToDoListAjout permet à l'utilisateur de rentrer les informations titre, note et heure.\\
    La classeToDoListDisplay est utilisée pour afficher la liste des informations données par l'utilisateur.\\
    ToDoListData permet la lecture et l'écriture dans un fichier des différentes informations.
  \section{Météo}
    ***La recherche de la météo est composée de meteodata.cpp/hpp, meteojour.cpp/hpp, meteomodule.cpp/hpp et meteoparam.cpp/hpp.\\
  \section{Actualité}
    ***La recherche d'actualité est composée de actudata.cpp/hpp, actumodule.cpp/hpp.\\
  \section{Horloge}
    L'horloge est composée de horlogemodule.cpp et horlogemodule.hpp. Nous avons utilisé un timer ce qui permet l'actualisation de la date et de l'heure toutes les 0.1s.\\
  \section{Convertisseur}
    Le convertisseur est composé de abstractmesureunite.cpp/hpp, celsius.cpp/hpp, convertmodule.cpp/hpp, fahrenheit.cpp/hpp, kelvin.cpp/hpp, kilometre.cpp/hpp, metre.cpp/hpp, miles.cpp/hpp, temperature.cpp/hpp, vitesse.cpp/hpp et volume.cpp/hpp."\\
    La classe AbstractMesureUnite est une classe abstraite qui sert de plateforfme de base pour les classes Vitesse et Température.\\
    Les classes Vitesse et Temperature servent de plateformes pour les classes Celsius, Fahrenheit, Kelvin, Kilometre, Metre, Miles et Volume.\\
    Les septs classes cités précédemment permettent de récupérer un réel, de la convertir dans la mesure voulu et d'afficher le réel obtenu après conversion.
\end{document}
\grid
\grid
