\documentclass[10pt]{beamer}
\usepackage[utf8]{inputenc}
\usepackage[T1]{fontenc}
\usepackage[french]{babel}
\usepackage{amsmath,minted,amsfonts,amssymb,tikz,graphicx,geometry,tkz-tab,pgfplots,float}
\usemintedstyle{tango}
\usetheme{UPPA2014}
\title{Projet POO, Dashboard}
\author{Romane Sallio \and Pierre Fontaine}
\institute{UPPA, Licence, Informatique}

\begin{document}

\begin{frame}
  \titlepage
\end{frame}

\begin{frame}
  \tableofcontents
\end{frame}

\section{introduction}

\begin{frame}
  \frametitle{Introduction}
  %\framesubtitle{Sous titre: objet}
  Dashboard est le projet que nous avons conçu en C++ avec le paradigme Objet.

  Nous avons utilisé le Framework QT pour développer l'interface utilisateur.

  Son objectif est simple : permettre à l'utilisateur d'avoir les éléments essentiels à porté de click.
\end{frame}

\section{Pourquoi QT}
\subsection{Des classes}
\begin{frame}
  QT est écrit en C++ et est implémenté selon le paradigme objet. Chacun des composants réfère à une classe particulière qui peut ou non dérivé d'une autre classe mère.
\end{frame}
  \subsection{JS Like}
    \begin{frame}
      Lors de l'utilisation d'un nouveau \emph{Framework}, une partie crucial du temps est concacré à l'étude du fonctionnement de celui ci. Il semblait évident qu'après avoir étudier le \emph{JavaScript}, le \emph{QT} qui partage la même philosophie de la gestion d'évènement \emph{(Async/Sync)} serait plus digeste.
    \end{frame}
\section{Spécifications techniques du code}
\subsection{Template}
\begin{frame}
  Utilisé dans \emph{List.h}\\

  Pourquoi ?
  \begin{enumerate}
    \item Créer une liste de n'importe quoi
    \item Container important
  \end{enumerate}
\end{frame}
\begin{frame}[fragile]
  \begin{minted}{c++}
    template <class T>
    class List{...}
  \end{minted}
\end{frame}
    \subsection{Heritage}
      \begin{frame}
        Objectif : rendre des composants plus spécifique.


      \end{frame}
    \subsection{Classe Complexe}
      \begin{frame}
        classe complexe ...
      \end{frame}
    \subsection{Classe Abstraite}
      \begin{frame}
        classe abstraite ...
      \end{frame}
\end{document}
